%\documentclass[a4paper,german,10pt]{tumarticle}
\documentclass[a4paper,english,10pt]{tumarticle}

\usepackage[utf8]{inputenc}
\usepackage{tumfonts}
\usepackage{tumlocale}
\usepackage{tumcmd}
\usepackage{booktabs}
\usepackage{eurosym}
\usepackage{enumitem}
\usepackage{subfig}
\usepackage[font=footnotesize,labelfont=bf]{caption}
\usepackage[hidelinks]{hyperref}
\usepackage{wrapfig}
\usepackage{pgfgantt}
\usepackage{adjustbox}
\usepackage{pdflscape}
\usepackage{hyperref}

\usetikzlibrary{decorations.text}

\usepackage{siunitx}
\sisetup{detect-weight=true, detect-family=true, per-mode=symbol}
\DeclareSIUnit\year{a}
\DeclareSIUnit{\EUR}{\text{EUR}}
\DeclareSIUnit{\week}{Woche}

\def\yes{\textcolor{TUMDarkerGreen}{\large\checkmark}}
\def\maybe{\textcolor{TUMOrange}{\Large$\mathbit\circ$}}
\def\no{\textcolor{TUMRed}{\Large\texttimes}}

% Set document title. If no language is supplied, the document language is
% assumed.

\linespread{1.32}
%\predisplaypenalty=10000000
%\floatingpenalty = 20000000
%\postdisplaypenalty = 2000000
\widowpenalty=15000
\setlength{\parindent}{0pt}



\begin{document}

\begin{center}
	\bfseries\Large Validation Tool for \texttt{libmoeprlnc}\\[.5\baselineskip]
\end{center}
\begin{center}
	\small Group 6\\ % Tobias Jülg and Ben Riegel % alphabetically sorted after last name
	\today
\end{center}


\setcounter{tocdepth}{1}
\renewcommand{\contentsname}{Anlagen}
%\begin{tableofcontents}
%\end{tableofcontents}

\renewcommand{\emph}[1]{%
	\textcolor{TUMBlue}{#1}%
}


% Motivation und Abstract
\renewcommand{\abstractname}{Abstract}
\begin{abstract}
\setlength{\parindent}{0pt}
\noindent%
\footnotesize

This paper proposes a testing framework for validating the \texttt{libmoeprlnc} library. Therefore,
it mainly uses randomly generated testing input.

\end{abstract}

\section{Introduction}

The \texttt{libmoeprlnc} library implements the encoding and decoding of blocks (generations) in
random linear network coding. It is based on the \texttt{libmoepgf} library, which in turn
implements the required Galois Field operations.

So far, \texttt{libmoeprlnc} does not have an adequate testing suit. Our goal is to develop a
testing environment for this library that mainly uses random test input in order to validate the
implementation. This document serves to outline our milestones and to provide a roadmap on how we
want to achieve them.

\section{Project Description}\label{sec:pd}

The overall idea of this project is to develop a testing framework. First, we want to address
easy-to-find bugs for cases where there have been changes to the coding- or math library. This will
be achieved by about five basic unit tests, which validate the implementation's most basic
functionalities by using handcrafted packets as input. For each test it will be checked if the
library can encode and decode the blocks as expected. Those tests will be referred  to as
'handcrafted tests' in the following

Since it is hard to find rare edge case bugs with fixed test cases, the core of our framework is a
more complex test which runs in a repeating fashion and tests completely random generated data for a
user-defined amount of iterations, potentially indefinitely many. With this method one can find even
very rare bug cases when run long enough. Also having completely random packets which are
independent of any network protocol will remove any kind of bias from the used testing input. In the
following, this core test will be referred to as the 'loop test'.

For this project, we define a test case to be failed if the decoder is able to decode a block but
the decoded block is not equal to the block which was originally given to the encoder.

At the end of the execution, the framework will provide the user with appropriate statistics which
summarize the test results. 

If an iteration of the loop test fails, the framework aborts and gives the user a dumb of the bytes
that were used as input. These can then be used for debugging. To ensure reproducibility, the random
bytes are generated using a seed. 

\section{Goals and Proposed Method}\label{sec:milestones}


In order to achieve the overall outcome as defined in Section \ref{sec:pd} we define the following
two main goals: The overall \textbf{testing framework} and the \textbf{statistics module}, where the
former can be subdivided into the \textbf{encoding/decoding workflow}, the implementation of the
\textbf{handcrafted tests} as well as the implementation of the \textbf{loop test}.

The base of our own testing environment should be a well known testing framework for \texttt{C}. We
decided to use \texttt{Check}\footnote{\url{https://libcheck.github.io/check/}}, because it is has a
rather simple interface which makes it easy to use and extendable.

It will be used for the execution of our handcrafted test as well as our loop test case. We plan to
use packets captured in a real-world scenario as a basis for the handcrafted tests since the purpose
of those tests is to cover common case bugs as explained in the section.

The loop test should be configurable by the following parameters which are passed over the command
line:
\begin{itemize}
    \setlength{\itemsep}{1pt}
	\setlength{\parskip}{0pt}
	\setlength{\parsep}{0pt}
    \item Number of iterations: How many packet blocks/generations should be sent
    \item Packet size range: The packet size will be picked from within this given range
    \item Packet generation size range
    \item Loss rate: Probability of a coded packet to be dropped on purpose
    \item Random seed to make the test reproducible
\end{itemize}

The second major goal of the project is the statistics module. Its purpose is to record interesting
statistics during the random data test. The following metrics should be recorded as a mean over all
previous random generations:
\begin{itemize}
    \setlength{\itemsep}{1pt}
	\setlength{\parskip}{0pt}
	\setlength{\parsep}{0pt}
    \item Percentage of successful generation decodings
    \item Amount of successful generations
    \item Theoretical expected coded packets needed for decoding vs actual needed packets
    \item Percentage of linear dependent packets
\end{itemize}
Whenever the decoding of a generation fails, a detailed log should be produced which contains the
whole generation's data, the iteration number after which the test failed and all parameters used
for the test configuration for reproducible. This log will probably be stored in the form of a text
or csv file in the user's working directory.

\section{Time Plan}

Figure \ref{timePlan} shows the approximated time plan for the proposed project and its milestones:
\begin{itemize}
    \setlength{\itemsep}{1pt}
	\setlength{\parskip}{0pt}
	\setlength{\parsep}{0pt}
    \item MS 1: Implementation of the encoding/decoding workflow with random packet loss and the
    thereafter check whether the packets match
    \item MS 2: Implementation of static packet test cases
    \item MS 3: Implementation of the random loop
    \item MS 4: Implementation of the statistic module
\end{itemize}


\begin{figure}[htb]
	\centering
	\begin{ganttchart}[
		hgrid,
		vgrid,
		x unit=0.15cm,
		y unit chart=0.5cm,
		time slot format=isodate
	]{2022-01-01}{2022-03-30}
		\gantttitlecalendar{month} \\
		\ganttbar{Proposal: Draft version}{2022-01-01}{2022-01-09}\\
		\ganttbar{Proposal: Final version}{2022-01-10}{2022-01-16}\\
		\ganttbar{MS 1}{2022-01-17}{2022-01-31}\\
		\ganttbar{MS 2}{2022-02-01}{2022-02-07}\\
		\ganttbar{MS 3}{2022-02-08}{2022-02-14}\\
		\ganttmilestone{Exam}{2022-02-21}\\
		\ganttbar{MS 4}{2022-02-22}{2022-03-07}
		% project deadline is not included because of missing date
		% presentation deadline is not included because of missing date
	\end{ganttchart}
	\caption{Project time plan}
	\label{timePlan}
\end{figure}

\end{document}

