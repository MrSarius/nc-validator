%\documentclass[a4paper,german,10pt]{tumarticle}
\documentclass[a4paper,english,10pt]{tumarticle}

\usepackage[utf8]{inputenc}
\usepackage{tumfonts}
\usepackage{tumlocale}
\usepackage{tumcmd}
\usepackage{booktabs}
\usepackage{eurosym}
\usepackage{enumitem}
\usepackage{subfig}
\usepackage[font=footnotesize,labelfont=bf]{caption}
\usepackage[hidelinks]{hyperref}
\usepackage{wrapfig}
\usepackage{pgfgantt}
\usepackage{adjustbox}
\usepackage{pdflscape}
\usepackage{hyperref}
\usepackage{listings}
\lstset{ 
  backgroundcolor=\color{white},
  breakatwhitespace=false,
  breaklines=true,
  frame=single,
  keepspaces=true,
  keywordstyle=\color{blue},
  language=Python,
  numbers=left,
  numbersep=5pt,
  rulecolor=\color{black},
  showspaces=false,
  showstringspaces=false,
  showtabs=false,
}

\usetikzlibrary{decorations.text}

\usepackage{siunitx}
\sisetup{detect-weight=true, detect-family=true, per-mode=symbol}
\DeclareSIUnit\year{a}
\DeclareSIUnit{\EUR}{\text{EUR}}
\DeclareSIUnit{\week}{Woche}

\def\yes{\textcolor{TUMDarkerGreen}{\large\checkmark}}
\def\maybe{\textcolor{TUMOrange}{\Large$\mathbit\circ$}}
\def\no{\textcolor{TUMRed}{\Large\texttimes}}

% Set document title. If no language is supplied, the document language is
% assumed.

\linespread{1.32}
%\predisplaypenalty=10000000
%\floatingpenalty = 20000000
%\postdisplaypenalty = 2000000
\widowpenalty=15000
\setlength{\parindent}{0pt}



\begin{document}

\begin{center}
	\bfseries\Large Validation Tool for \texttt{libmoeprlnc}\\[.5\baselineskip]
\end{center}
\begin{center}
	\small Group 6\\
	\today
\end{center}


\setcounter{tocdepth}{1}
\renewcommand{\contentsname}{Anlagen}
%\begin{tableofcontents}
%\end{tableofcontents}

\renewcommand{\emph}[1]{%
	\textcolor{TUMBlue}{#1}%
}


% Motivation und Abstract
\renewcommand{\abstractname}{Abstract}
\begin{abstract}
\setlength{\parindent}{0pt}
\noindent%
\footnotesize

This project proposes a validating tool for the \texttt{libmoeprlnc} library. 
This is achieved by calling the main interfaces \texttt{rlnc\_block\_add()}, 
\texttt{rlnc\_block\_encode()} / \texttt{rlnc\_block\_decode()} and \texttt{rlnc\_block\_get()} in a 
random order to find even very hard to find errors in one by running the tool through 
many iterations.\\
//TODO: Außerdem konnte die lib per statistischer auswertung validert werden\\
//TODO: Außerdem konnten die folgenden bugs gefunden werden-> Decode wirft 
keinen Fehler, wenn unlogischer Buffer übergeben wird (wollen wir das erwähnen)

\end{abstract}

\section{Introduction}

The \texttt{libmoeprlnc} library implements the encoding and decoding of generations in
random linear network coding. It is based on the \texttt{libmoepgf} library, which in turn
implements the required Galois Field operations.

The \texttt{libmoeprlnc} (also referred to as 'rlnc library' in the following) does not yet have an 
adequate testing suit. Our group proposes a validation tool for the rlnc library which mainly tries to 
find unexpected bugs by executing the libraries functions in a random order.
This document documents of the proposed tool, as well as the results found during our evaluation.

\section{Approach}\label{app}

\subsection{Brief Description of the RLNC library}\label{app:desc}
The following will briefly describe the functionality of the main interfaces, the RLNC library offers.
This is required to better understand the proposed approach by this project.
\\\\
First of all, the library uses a data structure called \texttt{rlnc\_block} which is the libraries representation of a generation.
Since our tool also has an abstraction of a generation, in the following the term "rlnc\_block" describes this libraries data structure and 
"generation" the abstraction of our tool. Besides this data structure, the library offers four main interfaces: 
\texttt{rlnc\_block\_add()}, \texttt{rlnc\_block\_encode()}, \texttt{rlnc\_block\_decode()} and \texttt{rlnc\_block\_get()}.

\begin{itemize}
    \item \underline{\texttt{rlnc\_block\_add(rlnc\_block\_t b, int pv, const uint8\_t *data, size\_t len)}}\\
    Reads the data of size \texttt{len} from \texttt{*data} and add it to the \texttt{rlnc\_block} \texttt{b}. Pivot (\texttt{pv}) describes the index of the data in the generation. 
    (TODO @Tobi ist das richtig mit dem pivot?). 

    \item \underline{\texttt{rlnc\_block\_encode(const rlnc\_block\_t b, uint8\_t *dst, size\_t maxlen, int flags)}}\\
    Encodes data from the \texttt{rlnc\_block} \texttt{b} pseudo randomly (TODO: drauf eingehen warum das 'pseudo' hier wichtig sein könnte) and returns the coded data at \texttt{*dst}. 
    \texttt{maxlen} describes the aligned length of the buffer \texttt{dst}. The alignment is chosen at the initialization of the \texttt{rlnc\_block}.
    
    \item \underline{\texttt{rlnc\_block\_decode(rlnc\_block\_t b, const uint8\_t *src, size\_t len)}}\\
    Receives coded data in \texttt{src} which is of length \texttt{len} and adds it to \texttt{rlnc\_block} \texttt{b}. 
    Then the \texttt{rlnc\_block} is decoded as far as possible. 
    Consequently, any linear dependencies that may have arisen are eliminated in this step. If the rank(b) should 
    be equal to the generation size after the execution of this method, the holder of \texttt{b} can now read the fully decoded generation.

    \item \underline{\texttt{rlnc\_block\_get(rlnc\_block\_t b, int pv, uint8\_t *dst, size\_t maxlen)}}\\
    If the requested data has already been decoded, the function returns the data of \texttt{rlnc\_block} \texttt{b} at index \texttt{pv} in buffer 
    \texttt{dst} of size \texttt{maxlen} (aligned).
  \end{itemize}

\subsection{TODO: HOW TO BUILD?}\label{app:build}
TODO: Dependencies + build-steps... Also wsl. einfach README reinkopieren. 

\subsection{TODO: OUR TOOL EXPLAINED}\label{app:expl}
TODO: Den Programmfluss etwas beschreiben. Wann passiert was und was davon passiert "random".

\subsection{Command Line Parameters}\label{app:cmd}
Our proposed tool accepts the following command line parameters:

\begin{itemize}
    \item -c, \texttt{-{}-}csv\_stats\\
    Print statistics to CSV file in current working directory.

    \item -f, \texttt{-{}-}field\_size=SIZE\\
    Set underlaying Galois Field size. (0: GF2, 1: GF4, 2: GF16, 3: GF256).

    \item -g, \texttt{-{}-}gen\_size=SIZE\\
    Set the generation size.

    \item -i, \texttt{-{}-}nr\_iterations=ITER\\
    Set the amount of test iterations

    \item -l, \texttt{-{}-}loss\_rate=LOSS\\
    Set probability with which a coded packet is lost during transmission.

    \item -p, \texttt{-{}-}pkt\_size=SIZE (TODO frame size??)\\
    Set the packet size.

    \item -s, \texttt{-{}-}seed=ADDR\\
    Set the seed which is used to generate random test input.
    \item -v, \texttt{-{}-}verbose

    Produce verbose output.
\end{itemize}

\section{Evaluation}\label{eval}
TODO: Hier grafiken und statistische Auswertungen.

\section{Validation of our Tool}\label{eval}
TODO: Jonas meinte ja dass wir irgendwie zumindest im Ansatz unser Tool auch validieren sollen. 

\section{Conclusion}\label{eval}
TODO: Hier könnten ggf. gefundene Bugs erwähnt werden.


\end{document}